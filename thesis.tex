%% NOTE: You may need to change the default paper size to letter in your
%% latex configuration tool... 
\documentclass[12pt,letterpaper]{report}
\usepackage{geometry}
\usepackage{natbib}
\usepackage{fancyhdr}
\usepackage{afterpage}
\usepackage{graphicx}
\usepackage{amsmath,amssymb,amsbsy}
\usepackage{dcolumn,array}
\usepackage{tocloft}
\usepackage{asudis}
\usepackage{acronym}

%-----------------------front matter
\pagenumbering{roman}
\title{One-Class Approaches for Object Extraction in Images}
\author{Salil Bhatra}

%% Information about yor committee & defense date. 
%% You must make sure this is correct and that you can ensure attendance from the 
%% required number of committee memebers. 

\degreeName{Master of Computing Studies}
\defensemonth{November}
\gradmonth{December}
\gradyear{2016}
\chair{Ashish Amresh}
\memberOne{John Femiani}
\memberTwo{Ajay Bansal}
\memberThree{Anshuman Razdan}
\memberFour{\ }




\begin{document}
\maketitle 

\doublespace

\begin{abstract}
This is my abstract
\end{abstract}
\dedicationpage{}
\acknowledgementpage{}
%{Enter your acknowledgement text here}
\tableofcontents
% This puts the word "Page" right justified above everything else.
\addtocontents{toc}{~\hfill Page\par}
% Asking LaTeX for a new page here guarantees that the LOF is on a separate page
% after the TOC ends.
\newpage
% Making the LOT and LOF "parts" rather than chapters gets them indented at
% level -1 according to the chart: top of page 4 of the document at
% ftp://tug.ctan.org/pub/tex-archive/macros/latex/contrib/tocloft/tocloft.pdf

% This gets the headers for the LOT right on the first page.  Subsequent pages
% are handled by the fancyhdr code in the asudis.sty file.
\addcontentsline{toc}{part}{LIST OF TABLES}
\renewcommand{\cftlabel}{Table}
\listoftables
\addtocontents{lot}{Table~\hfill Page \par}
\newpage
\addcontentsline{toc}{part}{LIST OF FIGURES}
\listoffigures
\addtocontents{lof}{Figure~\hfill Page \par}
\newpage
%\addcontentsline{toc}{part}{LIST OF SYMBOLS}
%\addtocontents{toc}{CHAPTER \par}
%\listofsymbols

% This gets the headers for the LOF right on the first page.  Subsequent pages
% are handled by the fancyhdr code in the asudis.sty file.
%% Acronyms used int the paper
%% Use \ac{UAV} or \acp{UAV} if it is plural. 
\acrodef{UAV}[UAV]{Unmanned Aerial Vehicle}


%-----------------------body
\doublespace
\pagenumbering{arabic}

\chapter{Introduction}

\section{Motivation}

%Study how prior art justifies feature extraction.
%look for examples and build an outline before writing this. 
% You MUST write, I will edit / suggest once you start the ball. 

%Basic idea: Lots of imagery (where is it coming from).  Some imporant applications (reconnaissancem, analysis, civil engineeringm mapmaking, simulation). Benefit from ability to find categories of object quickly.  Traditional classification involves binary decision learned by sampling target image patches and non-target image patches. We argue that magery is generally so convex is is not feesible to sample negative features adequately. 

\section{Problem Statement}

We address the problem of classification/ object extraction when only positive samples are labeled. Let us start by establishing some notation: Let ....

Then the problem can be stated as follows: we are \textbf{given} X and our \textbf{aim} is Y

\section{Objective}

The aim of this work is to evaluate methods that may solve the one-class extraction problem in the context of aerial imagery, particularly when applied to feature extraction problems such as the task of locating all image patches that contain vehicles in a large image. 


\section{Contributions}
%Three of them. Return to these as you get better/different than expected results.

We make the following contributions: 
\begin{itemize}
    \item (A contribution involving the application of this approach to cars?)
    \item Contribution 2
    \item Contribution 3 (always 3)
\end{itemize}

\chapter{Related Work}

% I recommend the use of jabref & a minf mapper like docear to organize work. 

The topic of one-class machine learning has been covered by the survey paper \cite{Khan2010, Khan2014}. 



% * More than a literature review
% * Organize related work - impose structure
% * Be clear as to how previous work being described relates to your own.
% * The reader should not be left wondering why you've described something!!
% * Identify opportunities for more research (i.e., your thesis) Are there unaddressed, or more important related topics?
% * After reading this chapter, one should understand the motivation for and importance of your thesis


\chapter {Theory}
% 8-20 pages

% Explain our approach (approaches) to one-class classification. Draw on the literature review you did
% What is it? 
% How is it generally accomplished?
% How does would we use it to find things in images?

% Why might one method work better?

\chapter{Implementation}
% 15-30 pages 

% ** DO NOT TALK BOUT YOUR CODE !! ****
% (you can put a link to a git repo in the footnotes if you must)

% Instead, talk about how you collected data. where it came from, 
% talk about how you plan to evaluate it.
% Talk about the specidfic classifiers you use
% intyroduce more notation ets. 


\chapter{Evaluation and Results}
%15-30 pages OR MORE
%figures and tables


%Focus on this FIRST, Priority 1

% present results AND this is VERY IMPORTANT, tell us how to interpret them. 
% Make sre your criteria judging are crystal clear

% give QUANTITATIVE tables etc.

%Also, point out examples (qualitative) where things work as expected, better than expected, or where they fail.
% Failure cases are very important, esp. with explanation of why they might have failed.

\section{Training Set Size vs Recall}
 
\begin{table}
  \centering
  \caption{Recall on 50\% of the labeled data when trained using increasing amounts of data.}
  \label{table:recall-vs-training-set-size}
  \begin{tabular}{cccccc} 
                                 & 10\% & 20\% & 30\% & 40\% & 50\% \\ 
      \hline 
      1C-SVM, \cite{chen2001one}  &      &      &      &      &      \\ 
      Method2                    &      &      &      &      &      \\ 
      Method3                    &      &      &      &      &      \\ 
      \hline 
  \end{tabular}
\end{table}





\chapter{Conclusion and Future Work}  
% 5-10 pages

% Do this LAST, it is least important and least difficult. 

% *  State what you've done and what you've found
% * Summarize contributions (achievements and impact)
% * Outline open issues/directions for future work





%-----------------------back matter
{\singlespace
% Making the references a "part" rather than a chapter gets it indented at
% level -1 according to the chart: top of page 4 of the document at
% ftp://tug.ctan.org/pub/tex-archive/macros/latex/contrib/tocloft/tocloft.pdf
\addcontentsline{toc}{part}{REFERENCES}
\bibliographystyle{asudis}
\bibliography{thesis}}
\renewcommand{\chaptername}{APPENDIX}
\addtocontents{toc}{APPENDIX \par}
\appendix
\chapter{RAW DATA}

\addcontentsline{toc}{part}{BIOGRAPHICAL}
\biographicalpage{}
\include{vita}
\end{document}